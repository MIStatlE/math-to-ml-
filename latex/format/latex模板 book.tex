\documentclass[10pt,openany]{book} % 可改 11pt,openright

% ==== Engine ====
\usepackage{ifxetex,ifluatex}
\usepackage{everypage}
\ifxetex\else\ifluatex\else
  \errmessage{Please compile with XeLaTeX or LuaLaTeX}
\fi\fi
% —— 让 plain 页式复用 fancy 的设置(章/部首页也有同样的页眉页脚)——
\makeatletter
\let\ps@plain\ps@fancy
\makeatother
% ==== Fonts & Page ====
\usepackage{ctex,fontspec,geometry,xcolor}
\geometry{paperwidth=6in,paperheight=9in, margin=0.8in} % 常见 6x9
\setmainfont{Times New Roman}
\IfFontExistsTF{Noto Serif CJK SC}{\setCJKmainfont{Noto Serif CJK SC}}{\setCJKmainfont{FandolSong}}
\linespread{1.35}
\pagecolor{white}\color{black}

% ==== Colors ====
\definecolor{brand}{HTML}{1A9D8F}
\definecolor{brandD}{HTML}{2A7F6F}
\definecolor{ink}{HTML}{1A1A1A}
\definecolor{keybg}{HTML}{E8F4F1}
\definecolor{keydark}{HTML}{2F5E58}
\definecolor{accent}{HTML}{FF6B6B}
\definecolor{accentD}{HTML}{D94F4F}
\definecolor{soft}{HTML}{F3F8F7}

% ==== Date macro ====
\newcommand{\padzero}[1]{\ifnum#1<10 0\fi\number#1}
\newcommand{\BrandYMD}{{\the\year·\padzero{\the\month}·\padzero{\the\day}}}

% ==== Left bands on every page ====
\usepackage{tikz,eso-pic}
\usetikzlibrary{positioning,calc}
\newcommand{\LeftBandA}{0.12in}
\newcommand{\LeftBandB}{0.28in}
\newcommand{\PutLeftBands}{%
  \begin{tikzpicture}[remember picture, overlay]
    \fill[keydark] ([xshift=0cm]current page.north west)
                   rectangle ([xshift=\LeftBandA]current page.south west);
    \fill[keybg]   ([xshift=\LeftBandA]current page.north west)
                   rectangle ([xshift=\LeftBandB]current page.south west);
  \end{tikzpicture}%
}
\AddEverypageHook{\PutLeftBands}

% ==== Header / Footer ====
\usepackage{fancyhdr}
\pagestyle{fancy}
\fancyhf{}
\renewcommand{\headrulewidth}{0.4pt}
\fancyhead[LE]{\nouppercase{\leftmark}}   % 偶数页:章标题
\fancyhead[RO]{\nouppercase{\rightmark}}  % 奇数页:节标题
\fancyfoot[LE,RO]{\textcolor{brand}{\thepage}}
\fancyfoot[LO,RE]{\textcolor{brand}{@MIStatlE|集中不等式 \seriesnum}}
% 封面样式
\fancypagestyle{coverstyle}{\fancyhf{}\renewcommand{\headrulewidth}{0pt}\renewcommand{\footrulewidth}{0pt}}

% ==== Series number parameter ====
\newcommand{\seriesnum}{I}

% ==== Titles (Part/Chapter/Section) ====
% —— Part 标题整块垂直居中 + 颜色/横线正确 —— 
\usepackage{titlesec}

\titleformat{\part}[display]
  {\filcenter\bfseries\color{keydark}}   % 颜色只作用于标题块内部
  {\Large Part \thepart}                 % 标签行(如:Part I)
  {0.6em}                                % 标签与标题的间距
  {\Huge}                                % 标题字号(开始标题)
  [\vspace{0.8em}{\color{brand}\rule{7cm}{1.2pt}}] % 结尾横线(在本地组里上色)

% 关键:用 titlespacing 的 before/after = \fill 来垂直居中整个块
\titlespacing*{\part}{0pt}{\fill}{\fill}
% Chapter
\titleformat{\chapter}[display]
  {\bfseries\color{keydark}}
  {\Large 第\thechapter 章}
  {0.4em}
  {\Huge}
  [\vspace{0.4em}{\color{brandD!45}\rule{6cm}{0.8pt}}]
\titlespacing*{\chapter}{0pt}{1.2em}{0.8em}

% Section / Subsection
\titleformat{\section}
  {\Large\bfseries\color{keydark}}
  {\thesection}{0.6em}{}
\titlespacing*{\section}{0pt}{0.9em}{0.5em}
\titleformat{\subsection}
  {\large\bfseries\color{brand}}
  {\thesubsection}{0.6em}{}
\titlespacing*{\subsection}{0pt}{0.8em}{0.4em}

% ==== (Optional) Chapter mini-TOC interface ====
% 如需每章小目录,取消注释以下三行,并在 \begin{document} 后调用 \dominitoc,
% 每个 \chapter 后面放 \minitoc 即可。
% \usepackage{minitoc}
% \setcounter{minitocdepth}{2}
% \setlength{\mtcindent}{12pt}

% ==== Boxes & Sidebars ====
\usepackage[most]{tcolorbox}
\tcbset{enhanced, boxrule=0.9pt, arc=3pt, left=10pt,right=10pt,top=10pt,bottom=10pt}
\newtcolorbox{KeyBox}{
  colback=keybg, colframe=brandD,
  colbacktitle=brandD, title=\textbf{\color{white}Key Idea},
  fonttitle=\bfseries\large, coltitle=white, breakable
}
\newtcolorbox{Takeaway}{
  colback=accent!12, colframe=accentD,
  colbacktitle=accentD, title=\textbf{\color{white}Takeaway},
  fonttitle=\bfseries\large, coltitle=white, breakable
}
\definecolor{exframe}{HTML}{6C5CE7}
\definecolor{exbg}{HTML}{F2E9FF}
\newtcolorbox{Example}[1][]{%
  enhanced, colback=exbg, colframe=exframe,
  coltitle=white, colbacktitle=exframe,
  title=\textbf{Example},
  fonttitle=\bfseries, boxrule=1.1pt, arc=4pt,
  left=10pt,right=10pt,top=10pt,bottom=10pt,
  attach boxed title to top left = {yshift=-2mm, xshift=2mm},
  breakable, #1
}
\usepackage{mdframed}
\newmdenv[
  topline=false, bottomline=false, rightline=false,
  linewidth=4pt, linecolor=brand, backgroundcolor=soft,
  innerleftmargin=10pt, innerrightmargin=10pt,
  skipabove=6pt, skipbelow=6pt
]{SideBar}

% ==== Math & Theorems ====
\usepackage{amsmath,amssymb,bm,amsthm}
\DeclareMathOperator{\Var}{Var}
\newcommand{\E}{\mathbb{E}}
\numberwithin{equation}{chapter}
\theoremstyle{plain}
\newtheorem{theorem}{定理}[chapter]
\newtheorem{lemma}[theorem]{引理}
\newtheorem{proposition}[theorem]{命题}
\newtheorem{corollary}[theorem]{推论}
\theoremstyle{definition}
\newtheorem{definition}[theorem]{定义}
\theoremstyle{remark}
\newtheorem{remark}[theorem]{注释}

% ==== Simple Cover Macro ====
% —— 极简封面(仅居中标题/副标题/日期)——
\newcommand{\MakeBookCoverUltraSimple}[3]{%
  % #1 主标题 #2 副标题 #3 标签(可空)
  \thispagestyle{coverstyle}%
  \vspace*{\fill}
  \begin{center}
    {\bfseries\fontsize{30}{36}\selectfont\color{keydark} #1\par}
    \vspace{0.45em}
    {\bfseries\fontsize{16}{22}\selectfont\color{ink!90} #2\par}
    \vspace{0.6em}
    \if\relax\detokenize{#3}\relax\else
      \vspace{0.2em}
      {\color{brandD}\fboxsep=6pt\colorbox{white}{\textbf{\textcolor{brandD}{#3}}}\par}
    \fi
    \vspace{0.8em}
    {\small\color{keydark!75}\BrandYMD}
  \end{center}
  \vspace*{\fill}
  \clearpage
}

% ==== Chapter Epigraph ====
\newcommand{\ChapterEpigraph}[2][]{%
  \par\vspace*{0.6em}
  \begin{flushright}
    \begin{minipage}{0.78\linewidth}\raggedleft
      {\itshape\color{ink!85}#2\par}%
      \if\relax\detokenize{#1}\relax\else
        \vspace{0.3em}{\color{ink!70}--- #1}\par
      \fi
    \end{minipage}
  \end{flushright}
  \vspace{0.6em}
}

% ==== Begin Document ====
\begin{document}
% \dominitoc % ← 若启用每章小目录,在此行取消注释

% 封面
\MakeBookCoverUltraSimple{集中不等式(示例书名)}{高维概率与机器学习理论中的浓缩现象}{上册 · 预印本}

% Frontmatter
\frontmatter
\thispagestyle{coverstyle}
\vspace*{3cm}
\begin{center}
{\Large\bfseries 致谢(示例)}\\[1.2em]
\parbox{0.75\linewidth}{\centering
感谢同行与读者的反馈……(此处放致谢内容占位)。}
\end{center}
\clearpage

\setcounter{tocdepth}{2} % 目录显示到 subsection
\tableofcontents
\clearpage

% Mainmatter
\mainmatter

\part{预备知识与工具}
\chapter{概率预备(示例)}
% \minitoc % ← 若启用 minitoc,这里放置
\ChapterEpigraph[Kolmogorov]{The theory of probability as a mathematical discipline can
and should be developed from axioms in exactly the same way as geometry and algebra.}

\begin{SideBar}
\textbf{阅读建议}:本章覆盖后续章节频繁使用的定义与工具。
\end{SideBar}

\section{Orlicz 范数与次高斯}
\begin{definition}[Orlicz 范数]
随机变量 $Z$ 的 $\psi_2$ 范数定义为
\[
\|Z\|_{\psi_2} := \inf\{s>0:\ \E e^{Z^2/s^2}\le 2\}.
\]
\end{definition}

\begin{lemma}[基本性质(示例)]
若 $X$ 次高斯,则 $X^2-\E X^2$ 次指数。
\end{lemma}

\begin{KeyBox}
\textbf{要点}:\(\psi_2\)-有界 $\Rightarrow$ 指数尾界;平方中心化 $\Rightarrow$ \(\psi_1\)-有界。
\end{KeyBox}

\section{Bernstein 不等式}
\begin{theorem}[Bernstein(示例)]
设 $Z_i$ 独立、$\|Z_i\|_{\psi_1}\le v$,则……
\end{theorem}

\begin{Takeaway}
小偏差 $\sim \|a\|_2$,大偏差 $\sim \|a\|_\infty$。
\end{Takeaway}

\part{主结果}
\chapter{Hanson--Wright 不等式(示例)}
% \minitoc
\section{定理与讨论}
\begin{Example}
这是一个示例环境,用于给出具体计算或反例。
\end{Example}

% Appendix
\appendix
\part{附录}
\chapter{符号与记号(示例)}
% \minitoc
\begin{itemize}
  \item \(\E\):数学期望
  \item \(\Var\):方差算子
\end{itemize}

% Backmatter
\backmatter
\begin{thebibliography}{9}
\bibitem{vershynin}
Vershynin, R. \emph{High-Dimensional Probability}. CUP, 2018.
\end{thebibliography}

\end{document}