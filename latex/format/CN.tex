\documentclass{article}
\usepackage{ctex,fontspec,geometry,xcolor}
\geometry{left=2.2cm,right=2.2cm,top=2.4cm,bottom=2.4cm}

% ---- 基础排版 ----
\setmainfont{Times New Roman}
\setCJKmainfont{FandolSong}
\usepackage{microtype,mathtools,bm,physics,enumitem}
\usepackage{titlesec,setspace,fancyhdr,hyperref,cleveref}
\setstretch{1.15}

% ---- 彩色章节 ----
\titleformat{\section}
  {\Large\bfseries\color{teal}}{\thesection.}{0.6em}{}[\titlerule]
% --- Algorithms ---
\usepackage[ruled,vlined,linesnumbered]{algorithm2e}
\renewcommand{\algorithmcfname}{Algorithm}
\SetKwInput{KwInput}{Input}
\SetKwInput{KwOutput}{Output}


% cleveref already loaded earlier (if not):
\usepackage[capitalize,nameinlink]{cleveref}

% ---- 定理环境 ----
\definecolor{shadecolor}{rgb}{0.90,0.95,1}
\definecolor{thmblue}{RGB}{35,116,190}
\usepackage{tcolorbox,thmtools}
\declaretheoremstyle[headfont=\bfseries\color{white},
  headformat=\COLORBOX{thmblue}{\color{white}\thmname{#1}~\thmnumber{#2}},
  shaded={bgcolor=shadecolor}]{thmbox}
\declaretheorem[style=thmbox,numberwithin=section]{theorem}
\declaretheorem[style=thmbox,sibling=theorem]{lemma}
\declaretheorem[style=thmbox,sibling=theorem]{definition}

% --- Extra boxes ---
% 深浅蓝 & 深浅红 统一
\newtcolorbox{analysisBox}{
  colback=blue!8!white, colframe=blue!60!black,
  coltitle=white, title=Analysis, fonttitle=\bfseries,
  rounded corners=2pt, boxrule=0.6pt, breakable
}
\newtcolorbox{remarkBox}{
  colback=red!8!white,  colframe=red!60!black,
  coltitle=white, title=Remark, fonttitle=\bfseries,
  rounded corners=2pt, boxrule=0.6pt, breakable
}


% ---- 页眉页脚 ----
\fancyhf{}\renewcommand{\headrulewidth}{0pt}
\fancyhead[L]{HD Probability}
\fancyhead[R]{\leftmark}
\fancyfoot[C]{\thepage}
\pagestyle{fancy}

\title{Suprema Expectation II (Enhanced)}
\author{}
\date{\today}
\begin{document}
\maketitle

\begin{abstract}
A pretty version of the previous notes …
\end{abstract}

\section{Introduction}
\subsection{subsection}
Text $\ldots$

\begin{definition}[ε–net]
Given a metric space $(T,d)$, a set $N\subset T$ is an $\varepsilon$–net if …
\end{definition}

\begin{lemma}
$N(T,d,\varepsilon)\le P(T,d,\varepsilon)$, etc.
\end{lemma}
\begin{algorithm}[H]
\KwInput{Data $D$, step size $\eta$}
\KwOutput{Optimal $x^\star$}
Initialize $x^{(0)}$\;
\For{$t=0$ \KwTo $T-1$}{
  $g_t \gets \nabla f\bigl(x^{(t)}\bigr)$\;
  $x^{(t+1)} \gets x^{(t)} - \eta\,g_t$\;
}
\caption{Gradient Descent}\label{alg:gd}
\end{algorithm}

\begin{analysisBox}
We prove \cref{thm:finite-approx} by decomposing the supremum into
projection error and net maximum, then apply Lemma~\ref{lem:finite-max}.
\end{analysisBox}

\begin{remarkBox}
Choosing $\varepsilon = 1/4$ balances the two terms and yields the
optimal $\sqrt{d}$ rate.
\end{remarkBox}

\printbibliography       % 若使用 biblatex
\end{document}
